%\documentclass[preprint,12pt]{elsarticle}
\documentclass[final,10pt]{elsarticle}

%% Use the option review to obtain double line spacing
%% \documentclass[preprint,review,12pt]{elsarticle}

%% Use the options 1p,twocolumn; 3p; 3p,twocolumn; 5p; or 5p,twocolumn
%% for a journal layout:
%% \documentclass[final,1p,times]{elsarticle}
%% \documentclass[final,1p,times,twocolumn]{elsarticle}
%% \documentclass[final,3p,times]{elsarticle}
%% \documentclass[final,3p,times,twocolumn]{elsarticle}
%% \documentclass[final,5p,times]{elsarticle}
%% \documentclass[final,5p,times,twocolumn]{elsarticle}

%% if you use PostScript figures in your article
%% use the graphics package for simple commands
\usepackage{graphics}
%% or use the graphicx package for more complicated commands
%%\usepackage{graphicx}
%% or use the epsfig package if you prefer to use the old commands
%% \usepackage{epsfig}

%% The amssymb package provides various useful mathematical symbols
\usepackage{amssymb}
\usepackage{amsmath}
%% The amsthm package provides extended theorem environments
%% \usepackage{amsthm}

\usepackage{url}\urlstyle{rm}
\usepackage{verbatim}



\RequirePackage{color}
\def\imagei{\centerline{\color[gray]{.75}\rule{\hsize}{4pc}}}%
\def\imageii{\centerline{\color[gray]{.75}\rule{4pc}{4pc}}}%

\newcommand{\vdag}{(v)^\dagger}
\newcommand{\us}{$\mu$s}
\newcommand{\imsc}{0.3}
%\newcommand{\imgext}{png}
\newcommand{\imgext}{eps}
\newcommand{\emaila}{scopatz@flash.uchicago.edu}




\journal{ICERM Position Paper}

\begin{document}

\begin{frontmatter}

%% Title, authors and addresses

%% use the tnoteref command within \title for footnotes;
%% use the tnotetext command for the associated footnote;
%% use the fnref command within \author or \address for footnotes;
%% use the fntext command for the associated footnote;
%% use the corref command within \author for corresponding author footnotes;
%% use the cortext command for the associated footnote;
%% use the ead command for the email address,
%% and the form \ead[url] for the home page:
%%
%% \title{Title\tnoteref{label1}}
%% \tnotetext[label1]{}
%% \author{Name\corref{cor1}\fnref{label2}}
%% \ead{email address}
%% \ead[url]{home page}
%% \fntext[label2]{}
%% \cortext[cor1]{}
%% \address{Address\fnref{label3}}
%% \fntext[label3]{}

\title{Passive Reproducibility: It's Not You, It's Me}


%% use optional labels to link authors explicitly to addresses:
%% \author[label1,label2]{<author name>}
%% \address[label1]{<address>}
%% \address[label2]{<address>}


\author[fc]{Anthony Scopatz} 
\ead{scopatz@flash.uchicago.edu}

%%%%%\email{\emaila}
\address[fc]{Flash Center for Computational Science, University of Chicago}


\begin{abstract}
Some Abstract text
\end{abstract}


\end{frontmatter}

Dear User, 

Uuugh, I don't know how to tell you this so I am just going to come out and
say it.  It's over.  There, bam, I'm sorry.  I would have liked to do this 
face-to-face or over the phone (but I am a computer).  So it goes.


I hope that we can still be friends.

Love Always, Computational Science


It should go without saying that reproducibility is a core tenent of science, 
technology, engineering, and mathematics (STEM).  Along with observation, logic, 
falsification, and disemination, reproducibility is part of the scaffolding that 
props up modern scientific inquiry.  Why then are we so bad at it when computers
are thrown in?

Perahps it is because computational tools are relatively new to human history.  
If this is the case, it is doubly perplexing then that 

%%%%%\bibliographystyle{spr-mp-nameyear-cnd}  
\bibliographystyle{model1-num-names}
%\bibliography{HEDP}

\end{document}


