%\documentclass[preprint,12pt]{elsarticle}
\documentclass[final,10pt]{elsarticle}

%% Use the option review to obtain double line spacing
%% \documentclass[preprint,review,12pt]{elsarticle}

%% Use the options 1p,twocolumn; 3p; 3p,twocolumn; 5p; or 5p,twocolumn
%% for a journal layout:
%% \documentclass[final,1p,times]{elsarticle}
%% \documentclass[final,1p,times,twocolumn]{elsarticle}
%% \documentclass[final,3p,times]{elsarticle}
%% \documentclass[final,3p,times,twocolumn]{elsarticle}
%% \documentclass[final,5p,times]{elsarticle}
%% \documentclass[final,5p,times,twocolumn]{elsarticle}

%% if you use PostScript figures in your article
%% use the graphics package for simple commands
\usepackage{graphics}
%% or use the graphicx package for more complicated commands
%%\usepackage{graphicx}
%% or use the epsfig package if you prefer to use the old commands
%% \usepackage{epsfig}

%% The amssymb package provides various useful mathematical symbols
\usepackage{amssymb}
\usepackage{amsmath}
%% The amsthm package provides extended theorem environments
%% \usepackage{amsthm}

\usepackage{url}\urlstyle{rm}
\usepackage{verbatim}



\RequirePackage{color}
\def\imagei{\centerline{\color[gray]{.75}\rule{\hsize}{4pc}}}%
\def\imageii{\centerline{\color[gray]{.75}\rule{4pc}{4pc}}}%

\newcommand{\vdag}{(v)^\dagger}
\newcommand{\us}{$\mu$s}
\newcommand{\imsc}{0.3}
%\newcommand{\imgext}{png}
\newcommand{\imgext}{eps}
\newcommand{\emaila}{scopatz@flash.uchicago.edu}




\journal{ICERM Position Paper}

\begin{document}

\begin{frontmatter}

%% Title, authors and addresses

%% use the tnoteref command within \title for footnotes;
%% use the tnotetext command for the associated footnote;
%% use the fnref command within \author or \address for footnotes;
%% use the fntext command for the associated footnote;
%% use the corref command within \author for corresponding author footnotes;
%% use the cortext command for the associated footnote;
%% use the ead command for the email address,
%% and the form \ead[url] for the home page:
%%
%% \title{Title\tnoteref{label1}}
%% \tnotetext[label1]{}
%% \author{Name\corref{cor1}\fnref{label2}}
%% \ead{email address}
%% \ead[url]{home page}
%% \fntext[label2]{}
%% \cortext[cor1]{}
%% \address{Address\fnref{label3}}
%% \fntext[label3]{}

\title{Passive Reproducibility: It's Not You, It's Me}


%% use optional labels to link authors explicitly to addresses:
%% \author[label1,label2]{<author name>}
%% \address[label1]{<address>}
%% \address[label2]{<address>}


\author[fc]{Anthony Scopatz} 
\ead{scopatz@flash.uchicago.edu}

%%%%%\email{\emaila}
\address[fc]{Flash Center for Computational Science, University of Chicago}


\begin{abstract}
Some Abstract text
\end{abstract}


\end{frontmatter}

Dear User, 

Uuugh, I don't know how to tell you this so I am just going to come out and
say it.  It's over.  There, bam, I'm sorry.  I would have liked to do this 
face-to-face or over the phone (but I am a computer).  So it goes.

I just need some time to work on myself.  You obviously were not paying enough 
attention to me.  You were always writing some paper or proceeding.  ``Publish or 
perish,'' I know.  Well you'd think that \emph{occassionally} you would be able 
to spend some time writing me!  I am publication also.  You write me, you copyright
me, and you put me out there in the world for others to read.  You would think that 
counts for something.  I have heard your arguments over and over and over that I 
can't get you a shiny faculty gig.  I think that is ridiculous.  Thousands of 
lines of code, a Ph.D., and a post-doc later and \emph{now} the review committee 
won't give me the time since epoch?!

Also there is this unatural obsession you have with the novelness of your work.
I understand that you want to do something new, something that distinguishes you.
You are smart. You have had some wins.  But look, science, technology, engineering, 
and mathematics are cummulative pursuits.  These things get built up over time.
Also, you are human.  You live and work in a community of people very similar to you.  
Most of them share this ridiculous prediliction with you.  I'll let you do the math
about a group of people with effectively the same training all trying to do something
differentially new at the same time.

You wanna do something really novel?  Run the same simulation twice and get the same
answer!  Bonus points for recording and saving for posterity exactly what you did to
get that answer.

OK, I am not saying that it isn't a noble pursuit to try to do something unique.  
But it isn't the only worthwhile pursuit.  I have needs too!  I feel like the quality
of my code base has decreased in the years that I have stood by you.  This is going
to come back to bite you.  $P=1$. I do not want to be around when it does.

Listen, I know that I am not perfect.  I am sorry.  Total reproducibility is hard.
It requires a lot of active, concetrated effort on your part.  With out anyone 
forcing you to do this, easy to understand that you dropped this whole long-term
reproducibility thing with all that you are juggling.  In my humble opinion, it is
kind of a failure of the education system that reproducibility wasn't beaten into you
to the point of self-regulation.  \emph{C'est la vie}, nothing we can do about that
now.

But it is not as if there are no tools out there whatsoever.  I know how much you
love labeling directories `v01', `v02', `v03', etc.  But honey, there are version 
control systems out there.  There are things called test suites out there. 

documentation, you like writing so much you 

Maybe I should just be invisible.

I am going to go find someone who respects me for who I am, maybe a software 
developer.  I hope that we can still be friends...

Love Always, Computational Science


%%%%%\bibliographystyle{spr-mp-nameyear-cnd}  
\bibliographystyle{model1-num-names}
%\bibliography{HEDP}

\end{document}


